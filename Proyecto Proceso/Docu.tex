% Generated by GrindEQ Word-to-LaTeX 
\documentclass{article} %%% use \documentstyle for old LaTeX compilers

\usepackage[english]{babel} %%% 'french', 'german', 'spanish', 'danish', etc.
\usepackage{amssymb}
\usepackage{amsmath}
\usepackage{txfonts}
\usepackage{mathdots}
\usepackage[classicReIm]{kpfonts}
\usepackage[dvips]{graphicx} %%% use 'pdftex' instead of 'dvips' for PDF output

% You can include more LaTeX packages here 


\begin{document}

%\selectlanguage{english} %%% remove comment delimiter ('%') and select language if required

\[2\] 



\noindent \includegraphics*[width=0.78in, height=0.96in, keepaspectratio=false]{image1}

\noindent \textbf{}

\noindent \textbf{UNIVERSIDAD DE GUAYAQUIL}

\noindent \textbf{FACULTAD DE CIENCIAS MATEM\'{A}TICAS Y F\'{I}SICAS}

\noindent \textbf{}

\noindent \textbf{PROYECTO SEGUNDO PARCIAL DE PROCESO DE SOFTWARE}

\noindent \textbf{``Aplicaci\'{o}n de dise\~{n}o y desarrollo en un software open source''}

\noindent NORMAS ISO 9001 DE EL SOFTWARE ``ERESTAURANT CSS''

\noindent 

\noindent \textbf{INTEGRANTES:}

\noindent \textbf{}

\begin{enumerate}
\item \textbf{ }JEFFERSON MINA AREVALO

\item  EILEEN VILLAMAR PLUAS

\item  JAMES CANTOS PINTO 
\end{enumerate}

\noindent 

\noindent 

\noindent \textbf{TUTOR:}

\begin{enumerate}
\item \textbf{ }ING.MIGUEL BOTTO .
\end{enumerate}

\noindent 

\noindent \textbf{GUAYAQUIL -- ECUADOR 2019}

\noindent \eject 

\noindent \textbf{INTRODUCCION A NORMAS 9001:2015}

\noindent Todo lo relativo a la calidad se est\'{a} convirtiendo en un tema cada vez m\'{a}s importante para las empresas debido a unos clientes cada vez m\'{a}s exigentes que obligan a \'{e}stas a adaptarse a las exigencias del mercado implantando Sistemas de Gesti\'{o}n de Calidad que satisfagan las expectativas y necesidades de los clientes adem\'{a}s de los requisitos reglamentarios y legales relativos a sus actividades. Por esta raz\'{o}n se han creado las normas ISO elaborada por la ``Organizaci\'{o}n Internacional para la Estandarizaci\'{o}n'' que especifica los requisitos para un sistema de gesti\'{o}n de la calidad que pueden utilizarse para su aplicaci\'{o}n interna por las organizaciones, para certificaci\'{o}n o con fines contractuales. Se centra en la eficacia del sistema de gesti\'{o}n de la calidad para satisfacer los requisitos del cliente.

\noindent Con la Implantaci\'{o}n de un Sistema de Gesti\'{o}n de Calidad las organizaciones pretenden consolidar los procesos que componen sus actividades y mejorar la eficacia de los mismos. Esto suma la importancia de que todos los desarrolladores y altos directivos de una empresa sepan sobre las normas de calidad para que el software o producto a desarrollarse pueda ser un trabajo con altos est\'{a}ndares de calidad. Estos est\'{a}ndares me certifican y me brindan la seguridad en mi sistema y una gran rentabilidad econ\'{o}mica, siempre va ser necesario que todos los ayudantes o trabajadores de una empresa sepan las normas ISO y su aplicaci\'{o}n a un software en especifico.

\noindent Una vez implantado debe comprobarse su correcto funcionamiento mediante una pre-auditor\'{i}a con personal correctamente cualificado para ello. Finalmente, se producir\'{a} otra pre-auditor\'{i}a en la que se realizar\'{a} una certificaci\'{o}n del sistema de calidad dise\~{n}ado. Una vez realizada esta, y habiendo obtenido un resultado positivo, la empresa est\'{a} en disposici\'{o}n de realizar ya una auditor\'{i}a de certificaci\'{o}n.

\noindent \textbf{}

\noindent \textbf{1.) Metodolog\'{i}a por usar}

\noindent \textbf{}

\noindent \textbf{        1.1) Manual de proceso}

\noindent En esta secci\'{o}n explicare como que se logro en el software elegido.

\noindent Todo naci\'{o} por una necesidad y aprendizaje de nuevas tecnolog\'{i}as.es un sistema desarrollado en JavaScript. Su fin era facilitar la experiencia   de un cliente al momento de visitar un restaurant, donde todo est\'{a} digitalizado y f\'{a}cil comprensi\'{o}n.

\noindent Para empezar el cliente tiene la opci\'{o}n de seleccionar su silla y ver su disponibilidad 

\noindent \includegraphics*[width=5.46in, height=3.35in, keepaspectratio=false, trim=2.02in 0.78in 4.34in 1.38in]{image2}

\noindent 

\noindent 

\noindent 

\noindent 

\noindent 

\noindent 

\noindent 

\noindent 

\noindent 

\noindent 

\noindent Una de las cosas que se quiere llevar acabo es la agilizaci\'{o}n de un proceso al momento de elegir    una mesa para el consumidor y una elecci\'{o}n m\'{a}s r\'{a}pida de su comida, esto evitar\'{i}a gastos en exceso por mas personal.

\noindent 

\noindent \textbf{ 1.2) Campo de aplicaci\'{o}n}

\noindent   Esto se lo utilizara en un restaurante con calificaci\'{o}n m\'{i}nima de 3 estrellas. Este sistema   debe implantarse a un sistema operativo que tenga la aplicaci\'{o}n JAVA y todas sus extensiones.

\noindent 

\noindent \includegraphics*[width=3.98in, height=4.74in, keepaspectratio=false]{image3}\textbf{ 1.3) Diagrama de Flujo }

\noindent Este diagrama de flujo permitir\'{a} entender mejor el funcionamiento del sistema. 

\noindent 

\noindent 

\noindent 

\noindent 

\noindent 

\noindent 

\noindent 

\noindent 

\noindent 

\noindent 

\noindent 

\noindent 

\noindent 

\noindent 

\noindent 

\noindent 

\noindent 

\noindent 

\noindent \textbf{  1.4) Procesos estrat\'{e}gicos o de mejora: }

\noindent Este sistema nos ayudara avanzar un nivel m\'{a}s sobre el mundo tecnol\'{o}gico, con futuras mejoras y actualizaciones podremos integrar m\'{a}s facilidades.. 

\noindent -Reservaciones Online. 

\noindent -Promocionar este sistema en otras compa\~{n}ias. 

\noindent - Auditor\'{i}as internas. o Procesos operativos o de prestaci\'{o}n del servicio: constituyen el core business de nuestra entidad y son los que tiene un impacto directo en la satisfacci\'{o}n de nuestros usuarios y otros grupos de inter\'{e}s.

\noindent  - Gesti\'{o}n de clientes. 

\noindent - Planificaci\'{o}n de actividades peri\'{o}dicas. 

\noindent - Planificaci\'{o}n de eventos. 

\noindent - Altas de Socios. 

\noindent - Reserva / alquiler instalaciones.

\noindent  - Bajas de socios.

\noindent \textbf{}

\noindent \textbf{}

\noindent \textbf{}

\noindent \textbf{}

\noindent \textbf{}

\noindent \textbf{}

\noindent \textbf{}

\noindent \textbf{}

\noindent \textbf{2.- Caso de estudio}

\noindent \textbf{     2.1) Introducci\'{o}n }

\noindent      ``ERESTAURANT CSS'' es un software libre para planificar un y ordenar con mucha mejor precisi\'{o}n el funcionamiento de un restaurante con tan solo la facilidad de un click. Con esto podemos dar mayor orden a un restaurante y reducci\'{o}n notable de personal esto generar\'{i}a mucho m\'{a}s ingreso para el gerente. ~Ayuda a automatizar y optimizar los procesos y tareas de tu restaurante, bar, caf\'{e} entre otros. Aqu\'{i} se registrar\'{a}n a los clientes, sus pedidos, sus ubicaciones y la valoraci\'{o}n a la comida. Es un sistema b\'{a}sico, pero a la vez es muy importante emplearlo para que no haya confusi\'{o}n ni malestares al momento de llegar un cliente.

\noindent \includegraphics*[width=5.54in, height=3.20in, keepaspectratio=false, trim=1.44in 0.41in 1.55in 1.19in]{image4} 

\noindent 

\noindent 

\noindent 

\noindent 

\noindent 

\noindent 

\noindent 

\noindent 

\noindent 

\noindent 

\noindent \textbf{}

\noindent \textbf{}

\noindent \textbf{}

\noindent \textbf{}

\noindent \textbf{2.2) Elementos del sistema}

\noindent Esta compuesto por varios m\'{o}dulos esta son las principales:

\begin{enumerate}
\item  Mesas 

\item  Pedido

\item  Facturar

\item  M\'{e}todo de Pago

\item  Configuraci\'{o}n de hora/fecha

\item  Usuario

\item  Administradores del sistema

\item  Empleados 
\end{enumerate}

\noindent 

\noindent 

\noindent \includegraphics*[width=4.86in, height=2.62in, keepaspectratio=false, trim=1.31in 0.94in 1.37in 0.68in]{image5}

\noindent 

\noindent 

\noindent 

\noindent 

\noindent 2.3) \textbf{Proceso Cliente-Software}

\noindent Seg\'{u}n las normas ISO 9001 para un buen uso del sistema software debe existir un amigable interacci\'{o}n Cliente-Software, entonces con esto da entender que para empezar el sistema debe tener una interfaz amigable con el usuario.

\begin{enumerate}
\item  Al momento de que el cliente quiera registrar a un usuario entonces el va tener acceso a las mesas disponibles.

\item  Una vez elegido su mesa podr\'{a} escribir su orden con todos pedidos extras que desee.

\item  El usuario podr\'{a} ver la sugerencia del chef y valoraciones del plato.

\item  Una vez elegido todo, podr\'{a} ver su factura y elegir consumidor final o datos.

\item  Por \'{u}ltimo podremos ver el m\'{e}todo de pago del usuario.

\item  Autom\'{a}ticamente se imprimir\'{a} una factura.
\end{enumerate}

\noindent \includegraphics*[width=6.75in, height=3.52in, keepaspectratio=false]{image6}

\noindent 

\noindent 

\noindent 

\noindent \textbf{2.4) Comunicaci\'{o}n Cliente-Software}

\noindent La comunicaci\'{o}n ente cliente -software garantiza que el sistema me esta brindando un soporte en caso de una emergencia por eso este sistema viene implementado estos requerimientos:

\begin{enumerate}
\item  Conexi\'{o}n con una base datos que validen los datos del usuario a consumir.

\item  Asistencias en soporte de una tarjeta de cr\'{e}dito o d\'{e}bito.

\item  Informaci\'{o}n de cada uno de los platos.

\item  Contraindicaci\'{o}n de los platos a vender.

\item  Una retroalimentaci\'{o}n de todo lo consumido, antes de facturar.
\end{enumerate}

\noindent \includegraphics*[width=6.67in, height=3.46in, keepaspectratio=false]{image7}

\noindent \textbf{}

\noindent 

\noindent 

\noindent 

\noindent 

\noindent 

\noindent 

\noindent \textbf{3) Dise\~{n}o y desarrollo}

\noindent \textbf{           3.1) Dise\~{n}o del Software}

\noindent El sistema esta orientado para la que las diversas interfaces integradas en el proyecto para que en conjunto se pueda integrar en un proceso exitoso y sobre todo de calidad comenzando de su dise\~{n}o grafico y dise\~{n}o de su arquitectura ya que esta separado por diversos m\'{o}dulos donde cada modulo esta comentado por cada funci\'{o}n que ejecuta el programa.

\noindent \textbf{    3.2) Dise\~{n}o de los datos}

\noindent Cada m\'{o}dulo tiene que relacionarse entre s\'{i}, debido que cada una de las acciones que se realice anteriormente, se tendr\'{a} que ver afectado a una decisi\'{o}n a futuro entonces, esto se logra que al momento de programar se utilizo funciones que tienen par\'{a}metros de una funci\'{o}n anterior.

\noindent Ej. del c\'{o}digo fuente.-

\begin{tabular}{|p{0.4in}|p{3.7in}|} \hline 
 \\ \hline 
/** \\ \hline 
 &  * 1. Avoid the WebKit bug in Android 4.0.* where \eqref{GrindEQ__2_} destroys native `audio` \\ \hline 
 &  *    and `video` controls. \\ \hline 
 &  * 2. Correct inability to style clickable `input` types in iOS. \\ \hline 
 &  * 3. Improve usability and consistency of cursor style between image-type \\ \hline 
 &  *    `input` and others. \\ \hline 
 &  */ \\ \hline 
 & \newline  \\ \hline 
 & button, \\ \hline 
 & html input[type="button"], /* 1 */ \\ \hline 
 & input[type="reset"], \\ \hline 
 & input[type="submit"] $\mathrm{\{}$ \\ \hline 
 &   -webkit-appearance: button; /* 2 */ \\ \hline 
 &   cursor: pointer; /* 3 */ \\ \hline 
 & $\mathrm{\}}$ \\ \hline 
 & \newline  \\ \hline 
 & /** \\ \hline 
 &  * Re-set default cursor for disabled elements. \\ \hline 
 &  */ \\ \hline 
 & \newline  \\ \hline 
 & button[disabled], \\ \hline 
 & html input[disabled] $\mathrm{\{}$ \\ \hline 
 &   cursor: default; \\ \hline 
 & $\mathrm{\}}$ \\ \hline 
 & \newline  \\ \hline 
 & /** \\ \hline 
 &  * Remove inner padding and border in Firefox 4+. \\ \hline 
 &  */ \\ \hline 
 & \newline  \\ \hline 
 & button::-moz-focus-inner, \\ \hline 
 & input::-moz-focus-inner $\mathrm{\{}$ \\ \hline 
 &   border: 0; \\ \hline 
 &   padding: 0; \\ \hline 
 & $\mathrm{\}}$ \\ \hline 
 & \newline  \\ \hline 
 & /** \\ \hline 
 &  * Address Firefox 4+ setting `line-height` on `input` using `!important` in \\ \hline 
 &  * the UA stylesheet. \\ \hline 
 &  */ \\ \hline 
 & \newline  \\ \hline 
 & input $\mathrm{\{}$ \\ \hline 
 &   line-height: normal; \\ \hline 
 & $\mathrm{\}}$ \\ \hline 
 & \newline  \\ \hline 
 & /** \\ \hline 
 &  * It's recommended that you don't attempt to style these elements. \\ \hline 
 &  * Firefox's implementation doesn't respect box-sizing, padding, or width. \\ \hline 
 &  * \\ \hline 
 &  * 1. Address box sizing set to `content-box` in IE 8/9/10. \\ \hline 
 &  * 2. Remove excess padding in IE 8/9/10. \\ \hline 
 &  */ \\ \hline 
 & \newline  \\ \hline 
 & input[type="checkbox"], \\ \hline 
 & input[type="radio"] $\mathrm{\{}$ \\ \hline 
 &   box-sizing: border-box; /* 1 */ \\ \hline 
 &   padding: 0; /* 2 */ \\ \hline 
 & $\mathrm{\}}$ \\ \hline 
 &  \\ \hline 
\end{tabular}



\noindent 

\noindent 

\noindent En este ejemplo se da a notar el dise\~{n}o de datos de como se relacionan cada uno de los datos que se utilizaron (en el ejemplo se explica como se relaciona el m\'{o}dulo de los botones).

\noindent 

\noindent \textbf{3.3) Dise\~{n}o Arquitect\'{o}nico}

\noindent El dise\~{n}o arquitect\'{o}nico es que el sistema se emplea con una base de datos MySQL y tambi\'{e}n esta empleada en HTML5. Donde a futuro se puede conseguir un servidor web e implementar en la intranet, el dise\~{n}o.

\noindent Ej.

\begin{tabular}{|p{0.6in}|p{4.0in}|} \hline 
$\mathrm{<}$!DOCTYPE html$\mathrm{>}$ \\ \hline 
 & $\mathrm{<}$html lang="es"$\mathrm{>}$ \\ \hline 
 & $\mathrm{<}$head$\mathrm{>}$ \\ \hline 
 &  $\mathrm{<}$meta charset="UTF-8"$\mathrm{>}$ \\ \hline 
 &  $\mathrm{<}$meta name="viewport" content="width=device-width, user-scalable=no, initial-scale=1.0, maximum-scale=1.0, minimum-scale=1.0"$\mathrm{>}$ \\ \hline 
 &  $\mathrm{<}$title$\mathrm{>}$Aplicacion Lacatedral$\mathrm{<}$/title$\mathrm{>}$ \\ \hline 
 &  $\mathrm{<}$link rel="stylesheet" href="css/normalize.css"$\mathrm{>}$ \\ \hline 
 &  $\mathrm{<}$link rel="stylesheet" href="css/app.css"$\mathrm{>}$ \\ \hline 
 & \newline  \\ \hline 
 &  $\mathrm{<}$script src="js/jquery-1.12.0.min.js"$\mathrm{>}$$\mathrm{<}$/script$\mathrm{>}$ \\ \hline 
 &  $\mathrm{<}$script src="js/app.js"$\mathrm{>}$$\mathrm{<}$/script$\mathrm{>}$ \\ \hline 
 & $\mathrm{<}$/head$\mathrm{>}$ \\ \hline 
 & $\mathrm{<}$body$\mathrm{>}$ \\ \hline 
 & \newline  \\ \hline 
 &  $\mathrm{<}$header class="header"$\mathrm{>}$ \\ \hline 
 &   $\mathrm{<}$h2 class="title--Header"$\mathrm{>}$ \\ \hline 
 &    Lacatedraldelpisco - App \\ \hline 
 &   $\mathrm{<}$/h2$\mathrm{>}$ \\ \hline 
 &  $\mathrm{<}$/header$\mathrm{>}$ \\ \hline 
 & \newline  \\ \hline 
 &  $\mathrm{<}$main$\mathrm{>}$ \\ \hline 
 &   $\mathrm{<}$header class="headerMain"$\mathrm{>}$ \\ \hline 
 &    $\mathrm{<}$div class="iz--HeaderMain"$\mathrm{>}$ \\ \hline 
 &     $\mathrm{<}$h2 class="title"$\mathrm{>}$ \\ \hline 
 &     \'{U}ltimos Pedidos - Pollos \\ \hline 
 &     $\mathrm{<}$/h2$\mathrm{>}$ \\ \hline 
 &    $\mathrm{<}$/div$\mathrm{>}$ \\ \hline 
 &    $\mathrm{<}$div class="de--HeaderMain"$\mathrm{>}$ \\ \hline 
 &     $\mathrm{<}$span class="title"$\mathrm{>}$ \\ \hline 
 &      \\ \hline 
 &     $\mathrm{<}$/span$\mathrm{>}$ \\ \hline 
 &    $\mathrm{<}$/div$\mathrm{>}$ \\ \hline 
 &   $\mathrm{<}$/header$\mathrm{>}$ \\ \hline 
 &   $\mathrm{<}$section class="pedidosMain"$\mathrm{>}$ \\ \hline 
 &    $\mathrm{<}$nav class="navPedidos"$\mathrm{>}$ \\ \hline 
 &     $\mathrm{<}$ul class="ul"$\mathrm{>}$ \\ \hline 
 &      $\mathrm{<}$li class="list"$\mathrm{>}$ \\ \hline 
 &       Mesa \\ \hline 
 &      $\mathrm{<}$/li$\mathrm{>}$ \\ \hline 
 &      $\mathrm{<}$li class="list"$\mathrm{>}$ \\ \hline 
 &       Pedido \\ \hline 
 &      $\mathrm{<}$/li$\mathrm{>}$ \\ \hline 
 &      $\mathrm{<}$li class="list"$\mathrm{>}$ \\ \hline 
 &       Hora \\ \hline 
 &      $\mathrm{<}$/li$\mathrm{>}$ \\ \hline 
 &      $\mathrm{<}$li class="list"$\mathrm{>}$ \\ \hline 
 &       Aprobar \\ \hline 
 &      $\mathrm{<}$/li$\mathrm{>}$ \\ \hline 
 &       \\ \hline 
 &     $\mathrm{<}$/ul$\mathrm{>}$ \\ \hline 
 &    $\mathrm{<}$/nav$\mathrm{>}$ \\ \hline 
 &    $\mathrm{<}$section class="content jsInsertMain"$\mathrm{>}$ \\ \hline 
 &      \\ \hline 
 &    $\mathrm{<}$/section$\mathrm{>}$ \\ \hline 
 & \newline  \\ \hline 
 &   $\mathrm{<}$/section$\mathrm{>}$ \\ \hline 
 &    \\ \hline 
 &  $\mathrm{<}$/main$\mathrm{>}$ \\ \hline 
 &   \\ \hline 
 & $\mathrm{<}$/body$\mathrm{>}$ \\ \hline 
 & $\mathrm{<}$/html$\mathrm{>}$ \\ \hline 
\end{tabular}



\noindent 

\noindent 

\noindent 

\noindent 

\noindent \textbf{3.4) Desarrollo}

\noindent \textbf{       3.4.1) Fase de actuaci\'{o}n}

\begin{enumerate}
\item \textbf{ Requisitos para los productos}. Uno de los mayores problemas en la organizaci\'{o}n era la falta de documentaci\'{o}n y gesti\'{o}n de la misma. En lugar de usar dibujos y especificaciones de condiciones, ten\'{i}an una muestra de cada producto y tomaron medidas de ellos para producir nuevos platos. El ingeniero que fue contratado con anterioridad entend\'{i}a la necesidad de sistematizaci\'{o}n de la documentaci\'{o}n y digitalizaci\'{o}n de los dibujos, pero al ser la \'{u}nica persona a encargada de la gesti\'{o}n nunca tuvo el tiempo suficiente para completarlo.

\item  \textbf{Proceso de producci\'{o}n}. Las personas que trabajaban en la producci\'{o}n eran muy h\'{a}biles, pero estaban mal dirigidas. Sin procedimientos e instrucciones precisas de trabajo nunca estaban seguros de lo que ten\'{i}an que hacer. A menudo se basaron en los deseos del director general en lugar de en las descripciones del trabajo. La creaci\'{o}n de un procedimiento documentado para la producci\'{o}n ayud\'{o} a definir todas las actividades en el proceso y as\'{i} mantener el proceso bajo control. Este problema tambi\'{e}n se trat\'{o} con el procedimiento de producci\'{o}n.
\end{enumerate}

\noindent 

\noindent \textbf{4)Fase de comprobaci\'{o}n}

\textbf{       4.1)Validaciones}

\begin{enumerate}
\item \textbf{ Monitoreo, medici\'{o}n, an\'{a}lisis y evaluaci\'{o}n}. No hab\'{i}a instrucciones estrictas y precisas sobre lo que debe ser monitoreado y medido, y cu\'{a}ndo. Los trabajadores se basaban la mayor\'{i}a en su experiencia. Al final del proceso de producci\'{o}n realizaban un simple control de calidad. Un trabajador examinaba los productos para identificar defectos y errores. Pero con frecuencia no sab\'{i}an d\'{o}nde buscar y qu\'{e} requisitos del producto eran necesarios satisfacer.

\item  \textbf{Manipulaci\'{o}n de las no conformidades y reclamaciones de los clientes}. 
\end{enumerate}

\noindent Las no conformidades identificadas nunca fueron analizadas y en lugar de eliminar la causa, simplemente se cre\'{o} un nuevo producto. Lo peor fue con quejas de los clientes, debido a que los productos fueron enviados a miles de kil\'{o}metros y no pod\'{i}an ser reparados. S\'{o}lo se sustituyeron, por lo que todas las quejas significaron grandes p\'{e}rdidas para la organizaci\'{o}n. La implementaci\'{o}n de un proceso de gesti\'{o}n de no conformidades y acciones correctivas fue muy beneficiosa para la organizaci\'{o}n. Ya que m\'{a}s tarde se descubrieron los fallos en el proceso de encolado y se lograron evitar m\'{a}s no conformidades.

\includegraphics*[width=6.66in, height=3.46in, keepaspectratio=false]{image8}

\noindent 

\noindent 

\noindent 

\noindent 

\noindent 

\noindent 

\noindent 

\noindent 

\noindent 


\end{document}

