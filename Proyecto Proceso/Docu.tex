% Generated by GrindEQ Word-to-LaTeX 
\documentclass{article} %%% use \documentstyle for old LaTeX compilers

\usepackage[english]{babel} %%% 'french', 'german', 'spanish', 'danish', etc.
\usepackage{amssymb}
\usepackage{amsmath}
\usepackage{txfonts}
\usepackage{mathdots}
\usepackage[classicReIm]{kpfonts}
\usepackage[dvips]{graphicx} %%% use 'pdftex' instead of 'dvips' for PDF output

% You can include more LaTeX packages here 


\begin{document}

%\selectlanguage{english} %%% remove comment delimiter ('%') and select language if required

\[2\] 



\noindent \includegraphics*[width=0.78in, height=0.96in, keepaspectratio=false]{image1}

\noindent \textbf{}

\noindent \textbf{UNIVERSIDAD DE GUAYAQUIL}

\noindent \textbf{FACULTAD DE CIENCIAS MATEM\'{A}TICAS Y F\'{I}SICAS}

\noindent \textbf{}

\noindent \textbf{PROYECTO SEGUNDO PARCIAL DE PROCESO DE SOFTWARE}

\noindent NORMAS ISO 9001 DE EL SOFTWARE ``DIMM SRI''

\noindent 

\noindent \textbf{INTEGRANTES:}

\noindent \textbf{}

\begin{enumerate}
\item \textbf{ }JEFFERSON MINA AREVALO

\item  EILEEN VILLAMAR PLUAS

\item  JAMES CANTOS PINTO 
\end{enumerate}

\noindent 

\noindent 

\noindent \textbf{TUTOR:}

\begin{enumerate}
\item \textbf{ }ING.MIGUEL BOTTO .
\end{enumerate}

\noindent 

\noindent \textbf{GUAYAQUIL -- ECUADOR 2019}

\noindent \eject 

\noindent \textbf{INTRODUCCION A NORMAS 9001:2015}

\noindent Todo lo relativo a la calidad se est\'{a} convirtiendo en un tema cada vez m\'{a}s importante para las empresas debido a unos clientes cada vez m\'{a}s exigentes que obligan a \'{e}stas a adaptarse a las exigencias del mercado implantando Sistemas de Gesti\'{o}n de Calidad que satisfagan las expectativas y necesidades de los clientes adem\'{a}s de los requisitos reglamentarios y legales relativos a sus actividades. Por esta raz\'{o}n se han creado las normas ISO elaborada por la ``Organizaci\'{o}n Internacional para la Estandarizaci\'{o}n'' que especifica los requisitos para un sistema de gesti\'{o}n de la calidad que pueden utilizarse para su aplicaci\'{o}n interna por las organizaciones, para certificaci\'{o}n o con fines contractuales. Se centra en la eficacia del sistema de gesti\'{o}n de la calidad para satisfacer los requisitos del cliente.

\noindent Con la Implantaci\'{o}n de un Sistema de Gesti\'{o}n de Calidad las organizaciones pretenden consolidar los procesos que componen sus actividades y mejorar la eficacia de los mismos. Esto suma la importancia de que todos los desarrolladores y altos directivos de una empresa sepan sobre las normas de calidad para que el software o producto a desarrollarse pueda ser un trabajo con altos est\'{a}ndares de calidad. Estos est\'{a}ndares me certifican y me brindan la seguridad en mi sistema y una gran rentabilidad econ\'{o}mica, siempre va ser necesario que todos los ayudantes o trabajadores de una empresa sepan las normas ISO y su aplicaci\'{o}n a un software en especifico.

\noindent Una vez implantado debe comprobarse su correcto funcionamiento mediante una pre-auditor\'{i}a con personal correctamente cualificado para ello. Finalmente, se producir\'{a} otra pre-auditor\'{i}a en la que se realizar\'{a} una certificaci\'{o}n del sistema de calidad dise\~{n}ado. Una vez realizada esta, y habiendo obtenido un resultado positivo, la empresa est\'{a} en disposici\'{o}n de realizar ya una auditor\'{i}a de certificaci\'{o}n.

\noindent \textbf{}

\noindent \textbf{1.) Metodolog\'{i}a por usar}

\noindent \textbf{}

\noindent \textbf{        1.1) Manual de proceso}

\noindent En esta secci\'{o}n explicare como que se logro en el software elegido.

\noindent Todo naci\'{o} por una necesidad y aprendizaje de nuevas tecnolog\'{i}as.es un sistema desarrollado en JavaScript. Su fin era facilitar la experiencia   de un cliente al momento de visitar un restaurant, donde todo est\'{a} digitalizado y f\'{a}cil comprensi\'{o}n.

\noindent Para empezar el cliente tiene la opci\'{o}n de seleccionar su silla y ver su disponibilidad 

\noindent \includegraphics*[width=5.46in, height=3.35in, keepaspectratio=false, trim=2.02in 0.78in 4.34in 1.38in]{image2}

\noindent 

\noindent 

\noindent 

\noindent 

\noindent 

\noindent 

\noindent 

\noindent 

\noindent 

\noindent 

\noindent Una de las cosas que se quiere llevar acabo es la agilizaci\'{o}n de un proceso al momento de elegir    una mesa para el consumidor y una elecci\'{o}n m\'{a}s r\'{a}pida de su comida, esto evitar\'{i}a gastos en exceso por mas personal.

\noindent 

\noindent \textbf{ 1.2) Campo de aplicaci\'{o}n}

\noindent   Esto se lo utilizara en un restaurante con calificaci\'{o}n m\'{i}nima de 3 estrellas. Este sistema   debe implantarse a un sistema operativo que tenga la aplicaci\'{o}n JAVA y todas sus extensiones.

\noindent 

\noindent \includegraphics*[width=3.98in, height=4.74in, keepaspectratio=false]{image3}\textbf{ 1.3) Diagrama de Flujo }

\noindent Este diagrama de flujo permitir\'{a} entender mejor el funcionamiento del sistema. 

\noindent 

\noindent 

\noindent 

\noindent 

\noindent 

\noindent 

\noindent 

\noindent 

\noindent 

\noindent 

\noindent 

\noindent 

\noindent 

\noindent 

\noindent 

\noindent 

\noindent 

\noindent 

\noindent \textbf{  1.4) Procesos estrat\'{e}gicos o de mejora: }

\noindent Este sistema nos ayudara avanzar un nivel m\'{a}s sobre el mundo tecnol\'{o}gico, con futuras mejoras y actualizaciones podremos integrar m\'{a}s facilidades.. 

\noindent -Reservaciones Online. 

\noindent -Promocionar este sistema en otras compa\~{n}ias. 

\noindent - Auditor\'{i}as internas. o Procesos operativos o de prestaci\'{o}n del servicio: constituyen el core business de nuestra entidad y son los que tiene un impacto directo en la satisfacci\'{o}n de nuestros usuarios y otros grupos de inter\'{e}s.

\noindent  - Gesti\'{o}n de clientes. 

\noindent - Planificaci\'{o}n de actividades peri\'{o}dicas. 

\noindent - Planificaci\'{o}n de eventos. 

\noindent - Altas de Socios. 

\noindent - Reserva / alquiler instalaciones.

\noindent  - Bajas de socios.

\noindent \textbf{}

\noindent \textbf{}

\noindent \textbf{}

\noindent \textbf{}

\noindent \textbf{}

\noindent \textbf{}

\noindent \textbf{}

\noindent \textbf{}

\noindent \textbf{2.- Caso de estudio}

\noindent \textbf{     2.1) Introducci\'{o}n }

\noindent      ``ERESTAURANT CSS'' es un software libre para planificar un y ordenar con mucha mejor precisi\'{o}n el funcionamiento de un restaurante con tan solo la facilidad de un click. Con esto podemos dar mayor orden a un restaurante y reducci\'{o}n notable de personal esto generar\'{i}a mucho m\'{a}s ingreso para el gerente. ~Ayuda a automatizar y optimizar los procesos y tareas de tu restaurante, bar, caf\'{e} entre otros. Aqu\'{i} se registrar\'{a}n a los clientes, sus pedidos, sus ubicaciones y la valoraci\'{o}n a la comida. Es un sistema b\'{a}sico, pero a la vez es muy importante emplearlo para que no haya confusi\'{o}n ni malestares al momento de llegar un cliente.

\noindent \includegraphics*[width=5.54in, height=3.20in, keepaspectratio=false, trim=1.44in 0.41in 1.55in 1.19in]{image4} 

\noindent 

\noindent 

\noindent 

\noindent 

\noindent 

\noindent 

\noindent 

\noindent 

\noindent 

\noindent 

\noindent \textbf{}

\noindent \textbf{}

\noindent \textbf{}

\noindent \textbf{}

\noindent \textbf{2.2) Elementos del sistema}

\noindent Esta compuesto por varios m\'{o}dulos esta son las principales:

\begin{enumerate}
\item  Mesas 

\item  Pedido

\item  Facturar

\item  M\'{e}todo de Pago

\item  Configuraci\'{o}n de hora/fecha

\item  Usuario

\item  Administradores del sistema

\item  Empleados 
\end{enumerate}

\noindent 

\noindent 

\noindent \includegraphics*[width=4.86in, height=2.62in, keepaspectratio=false, trim=1.31in 0.94in 1.37in 0.68in]{image5}

\noindent 

\noindent 

\noindent 

\noindent 

\noindent 2.3) \textbf{Proceso Cliente-Software}

\noindent Seg\'{u}n las normas ISO 9001 para un buen uso del sistema software debe existir un amigable interacci\'{o}n Cliente-Software, entonces con esto da entender que para empezar el sistema debe tener una interfaz amigable con el usuario.

\begin{enumerate}
\item  Al momento de que el cliente quiera registrar a un usuario entonces el va tener acceso a las mesas disponibles.

\item  Una vez elegido su mesa podr\'{a} escribir su orden con todos pedidos extras que desee.

\item  El usuario podr\'{a} ver la sugerencia del chef y valoraciones del plato.

\item  Una vez elegido todo, podr\'{a} ver su factura y elegir consumidor final o datos.

\item  Por \'{u}ltimo podremos ver el m\'{e}todo de pago del usuario.

\item  Autom\'{a}ticamente se imprimir\'{a} una factura.
\end{enumerate}

\noindent \includegraphics*[width=6.75in, height=3.52in, keepaspectratio=false]{image6}

\noindent 

\noindent 

\noindent 

\noindent \textbf{2.4) Comunicaci\'{o}n Cliente-Software}

\noindent La comunicaci\'{o}n ente cliente -software garantiza que el sistema me esta brindando un soporte en caso de una emergencia por eso este sistema viene implementado estos requerimientos:

\begin{enumerate}
\item  Conexi\'{o}n con una base datos que validen los datos del usuario a consumir.

\item  Asistencias en soporte de una tarjeta de cr\'{e}dito o d\'{e}bito.

\item  Informaci\'{o}n de cada uno de los platos.

\item  Contraindicaci\'{o}n de los platos a vender.

\item  Una retroalimentaci\'{o}n de todo lo consumido, antes de facturar.
\end{enumerate}

\includegraphics*[width=6.67in, height=3.46in, keepaspectratio=false]{image7}

\noindent \textbf{}

\noindent 

\noindent 

\noindent 

\noindent 

\noindent 

\noindent 

\noindent 

\noindent 

\noindent 

\noindent 

\noindent 

\noindent 


\end{document}

