% Generated by GrindEQ Word-to-LaTeX 
\documentclass{article} %%% use \documentstyle for old LaTeX compilers

\usepackage[english]{babel} %%% 'french', 'german', 'spanish', 'danish', etc.
\usepackage{amssymb}
\usepackage{amsmath}
\usepackage{txfonts}
\usepackage{mathdots}
\usepackage[classicReIm]{kpfonts}
\usepackage[dvips]{graphicx} %%% use 'pdftex' instead of 'dvips' for PDF output

% You can include more LaTeX packages here 


\begin{document}

%\selectlanguage{english} %%% remove comment delimiter ('%') and select language if required

\[2\] 



\noindent \includegraphics*[width=0.78in, height=0.96in, keepaspectratio=false]{image1}

\noindent \textbf{}

\noindent \textbf{UNIVERSIDAD DE GUAYAQUIL}

\noindent \textbf{FACULTAD DE CIENCIAS MATEM\'{A}TICAS Y F\'{I}SICAS}

\noindent \textbf{}

\noindent \textbf{PROYECTO SEGUNDO PARCIAL DE PROCESO DE SOFTWARE}

\noindent NORMAS ISO 9001 DE EL SOFTWARE ``DIMM SRI''

\noindent 

\noindent \textbf{INTEGRANTES:}

\noindent \textbf{}

\begin{enumerate}
\item \textbf{ }JEFFERSON MINA AREVALO

\item  EILEEN VILLAMAR PLUAS

\item  JAMES CANTOS PINTO 
\end{enumerate}

\noindent 

\noindent 

\noindent \textbf{TUTOR:}

\begin{enumerate}
\item \textbf{ }ING.MIGUEL BOTTO .
\end{enumerate}

\noindent 

\noindent \textbf{GUAYAQUIL -- ECUADOR 2019}

\noindent \eject 

\noindent \textbf{INTRODUCCION A NORMAS 9001:2015}

\noindent Todo lo relativo a la calidad se est\'{a} convirtiendo en un tema cada vez m\'{a}s importante para las empresas debido a unos clientes cada vez m\'{a}s exigentes que obligan a \'{e}stas a adaptarse a las exigencias del mercado implantando Sistemas de Gesti\'{o}n de Calidad que satisfagan las expectativas y necesidades de los clientes adem\'{a}s de los requisitos reglamentarios y legales relativos a sus actividades. Por esta raz\'{o}n se han creado las normas ISO elaborada por la ``Organizaci\'{o}n Internacional para la Estandarizaci\'{o}n'' que especifica los requisitos para un sistema de gesti\'{o}n de la calidad que pueden utilizarse para su aplicaci\'{o}n interna por las organizaciones, para certificaci\'{o}n o con fines contractuales. Se centra en la eficacia del sistema de gesti\'{o}n de la calidad para satisfacer los requisitos del cliente.

\noindent Con la Implantaci\'{o}n de un Sistema de Gesti\'{o}n de Calidad las organizaciones pretenden consolidar los procesos que componen sus actividades y mejorar la eficacia de los mismos. Esto suma la importancia de que todos los desarrolladores y altos directivos de una empresa sepan sobre las normas de calidad para que el software o producto a desarrollarse pueda ser un trabajo con altos est\'{a}ndares de calidad. Estos est\'{a}ndares me certifican y me brindan la seguridad en mi sistema y una gran rentabilidad econ\'{o}mica, siempre va ser necesario que todos los ayudantes o trabajadores de una empresa sepan las normas ISO y su aplicaci\'{o}n a un software en especifico.

\noindent Una vez implantado debe comprobarse su correcto funcionamiento mediante una pre-auditor\'{i}a con personal correctamente cualificado para ello. Finalmente, se producir\'{a} otra pre-auditor\'{i}a en la que se realizar\'{a} una certificaci\'{o}n del sistema de calidad dise\~{n}ado. Una vez realizada esta, y habiendo obtenido un resultado positivo, la empresa est\'{a} en disposici\'{o}n de realizar ya una auditor\'{i}a de certificaci\'{o}n.

\noindent \textbf{}

\noindent \textbf{1.) Metodolog\'{i}a a usar}

\noindent \textbf{        1.1) Manual de proceso}

\noindent En esta secci\'{o}n explicare como que se logro en el software elegido.

\noindent Todo naci\'{o} por una necesidad y aprendizaje de nuevas tecnolog\'{i}as.es un sistema desarrollado en JavaScript. Su fin era facilitar la experiencia   de un cliente al momento de visitar un restaurant, donde todo est\'{a} digitalizado y f\'{a}cil comprensi\'{o}n.

\noindent Para empezar el cliente tiene la opci\'{o}n de seleccionar su silla y ver su disponibilidad 

\noindent \includegraphics*[width=5.46in, height=3.35in, keepaspectratio=false, trim=2.02in 0.78in 4.34in 1.38in]{image2}

\noindent 

\noindent 

\noindent 

\noindent 

\noindent 

\noindent 

\noindent 

\noindent 

\noindent 

\noindent Una de las cosas que se quiere llevar acabo es la agilizaci\'{o}n de un proceso al momento de elegir    una mesa para el consumidor y una elecci\'{o}n m\'{a}s r\'{a}pida de su comida, esto evitar\'{i}a gastos en exceso por mas personal.

\noindent 

\noindent 

\noindent 

\noindent \textbf{ 1.2) Campo de aplicaci\'{o}n}

\noindent   Esto se lo utilizara en un restaurante con calificaci\'{o}n m\'{i}nima de 3 estrellas. Este sistema   debe implantarse a un sistema operativo que tenga la aplicaci\'{o}n JAVA y todos sus extensiones.

\noindent 

\noindent \includegraphics*[width=3.98in, height=4.74in, keepaspectratio=false]{image3}\textbf{ 1.3) Diagrama de Flujo }

\noindent Este diagrama de flujo permitir\'{a} entender mejor el funcionamiento del sistema. 

\noindent 

\noindent 

\noindent 

\noindent 

\noindent 

\noindent 

\noindent 

\noindent 

\noindent 

\noindent 

\noindent 

\noindent 

\noindent 

\noindent 

\noindent 

\noindent 

\noindent 

\noindent 

\noindent \textbf{2.- Caso de estudio}

\noindent      ``ERESTAURANT CSS'' es un software libre para planificar un y ordenar con mucha mejor precisi\'{o}n el funcionamiento de un restaurante con tan solo la facilidad de un click. Con esto podemos dar mayor orden a un restaurante y reducci\'{o}n notable de personal esto generar\'{i}a mucho m\'{a}s ingreso para el gerente. ~Ayuda a automatizar y optimizar los procesos y tareas de tu restaurante, bar, caf\'{e} entre otros. Aqu\'{i} se registrar\'{a}n a los clientes, sus pedidos, sus ubicaciones y la valoraci\'{o}n a la comida. Es un sistema b\'{a}sico, pero a la vez es muy importante emplearlo para que no haya confusi\'{o}n ni malestares al momento de llegar un cliente. 

\noindent 


\end{document}

